%%%%%%%%%%%%%%%%%%%%%%%%%%%%%%%%%%%%%%%%%%%%%%%%%%%%%%%%%%%%%%%%%%
%%%%%%%% ICML 2015 EXAMPLE LATEX SUBMISSION FILE %%%%%%%%%%%%%%%%%
%%%%%%%%%%%%%%%%%%%%%%%%%%%%%%%%%%%%%%%%%%%%%%%%%%%%%%%%%%%%%%%%%%

% Use the following line _only_ if you're still using LaTeX 2.09.
%\documentstyle[icml2015,epsf,natbib]{article}
% If you rely on Latex2e packages, like most moden people use this:
\documentclass{article}

% use Times
\usepackage{times}
% For figures
\usepackage{graphicx} % more modern
%\usepackage{epsfig} % less modern
\usepackage{subfigure} 

% For citations
\usepackage{natbib}

% For algorithms
\usepackage{algorithm}
\usepackage{algorithmic}

% As of 2011, we use the hyperref package to produce hyperlinks in the
% resulting PDF.  If this breaks your system, please commend out the
% following usepackage line and replace \usepackage{icml2015} with
% \usepackage[nohyperref]{icml2015} above.
\usepackage{hyperref}

% Packages hyperref and algorithmic misbehave sometimes.  We can fix
% this with the following command.
\newcommand{\theHalgorithm}{\arabic{algorithm}}

% Employ the following version of the ``usepackage'' statement for
% submitting the draft version of the paper for review.  This will set
% the note in the first column to ``Under review.  Do not distribute.''
\usepackage[accepted]{icml2015} 


% The \icmltitle you define below is probably too long as a header.
% Therefore, a short form for the running title is supplied here:
\icmltitlerunning{Homework1}

\begin{document} 

\twocolumn[
\icmltitle{Homework1}

% You may provide any keywords that you 
% find helpful for describing your paper; these are used to populate 
% the "keywords" metadata in the PDF but will not be shown in the document
\icmlkeywords{boring formatting information, machine learning, ICML}

%\vskip 0.3in
]
%\abstract{The following paper outlines our approach to }

\section{Implement Gradient Descent}
Our gradient descent code is written in python and takes as parameters the initial guess, $x_0$, the step size, $\gamma$, and the convergence threshold, $\epsilon$.  The convergence criteria that we use to terminate is when the difference between successive ``guesses'' (call these $x_n$ and $x_{n+1}$ for some arbitrary $n$) is less than $\epsilon$, i.e. $|x_{n+1} -x_n | < \epsilon$.  Since we are only working with scalar functions, this will suffice.  If we were working with vector values functions, we could instead use the two-norm.  

In order to test our gradient descent, we experimented with a few different functions.  We used a quadratic function: $f(X) = \sum_{i=1}^n x_i^2$, the Rosenbrock function: $f(x_1,x_2) = (1-x_1)^2+100(x_2-x_1^2)^2$ (as referenced in the Wikipedia page for gradient descent), and a few non convex functions to test gradient descent with multiple minima.  

The choice of starting guess, $x_0$, the step size, $\gamma$ and the convergence threshold are very import to yield the ``correct'' solution.  In order to be confident that the algorithm has reached a solution, it's important to pick a small $\epsilon$.  For most of our testing, we used $\epsilon = 10^{-6}$.  However, picking too small of an epsilon can be computationally expensive: causing the algorithm to perform many more iterations that necessary to produce a ``confident'' solution.  The choice of starting guess, $x_0$ is also interesting.  In order to guarantee convergence, one needs to pick an $x_0$ relatively close to the true solution.  (In the case of more than one minima, the algorithm will converge to the minima closest to the starting point).  We found that However, we found the most important criteria was the $\gamma$ parameter.  We usually started with $\gamma=1$ and decreased it's value until we found a good fit.  If $\gamma$ is to large, the algorithm could miss the minima by jumping back and forth between the same points.  If $\gamma$ is too small, the algorithm may not converge as quickly...Mention numerical errors.  

We also wrote code to approximate the gradient of a function numerically using central differences.  We verified its behavior by comparing the analytical and numerical gradients in a testing suite.  We used the python unittest framework and the assertAlmostEquals function to guarantee that all approximations were within 7 digits of accuracy.  We tested the gradient on the same functions we used to test our gradient descent, and also a few common polynomial functions. 



\section{Linear Basis Function Regression}

(2.4) In this instance, we used a polynomial basis function.  If instead the basis functions were sine functions, $\phi_1(x)=\sin(2\pi x),...,\phi_M(x)=\sin(2\pi Mx)$ we would expect that these basis functions would insure an even better fit for the data.  In fact, testing this on our data set, we found that the SEE was much smaller (insert example).  However, the sine basis functions are not always a good fit.  If we did not know that the data was generated from $\sin(2\pi x)$, using an oscillating basis function like the sine can cause overfitting (like we saw in the polynomial basis case of $M=9$ on the curve fitting data set).  Also, as alluded to, using a sine basis would force our data to oscillate, which may not be a good choice if our data doesn't have that shape.  Examples include $x^3$, which has a saddle point and does not oscillate, or something with an exponential growth (like $e^x$)

\section{Ridge Regression} 

\section{Generalizations}
\subsection{Outlier and LAD}
\subsection{Sparsity and LASSO}

%% FIGURE EXAMPLE %%
\begin{figure}[ht]
\vskip 0.2in
\begin{center}
\centerline{\includegraphics[width=\columnwidth]{icml_numpapers}}
\caption{Historical locations and number of accepted papers for International
  Machine Learning Conferences (ICML 1993 -- ICML 2008) and
  International Workshops on Machine Learning (ML 1988 -- ML
  1992). At the time this figure was produced, the number of
  accepted papers for ICML 2008 was unknown and instead estimated.}
\label{icml-historical}
\end{center}
\vskip -0.2in
\end{figure} 

\subsection{Figures}
 
You may want to include figures in the paper to help readers visualize
your approach and your results. Such artwork should be centered,
legible, and separated from the text. Lines should be dark and at
least 0.5~points thick for purposes of reproduction, and text should
not appear on a gray background.

Label all distinct components of each figure. If the figure takes the
form of a graph, then give a name for each axis and include a legend
that briefly describes each curve. Do not include a title inside the
figure; instead, the caption should serve this function.

Number figures sequentially, placing the figure number and caption
{\it after\/} the graphics, with at least 0.1~inches of space before
the caption and 0.1~inches after it, as in
Figure~\ref{icml-historical}.  The figure caption should be set in
9~point type and centered unless it runs two or more lines, in which
case it should be flush left.  You may float figures to the top or
bottom of a column, and you may set wide figures across both columns
(use the environment {\tt figure*} in \LaTeX), but always place
two-column figures at the top or bottom of the page.

\subsection{Algorithms}

If you are using \LaTeX, please use the ``algorithm'' and ``algorithmic'' 
environments to format pseudocode. These require 
the corresponding stylefiles, algorithm.sty and 
algorithmic.sty, which are supplied with this package. 
Algorithm~\ref{alg:example} shows an example. 

\begin{algorithm}[tb]
   \caption{Bubble Sort}
   \label{alg:example}
\begin{algorithmic}
   \STATE {\bfseries Input:} data $x_i$, size $m$
   \REPEAT
   \STATE Initialize $noChange = true$.
   \FOR{$i=1$ {\bfseries to} $m-1$}
   \IF{$x_i > x_{i+1}$} 
   \STATE Swap $x_i$ and $x_{i+1}$
   \STATE $noChange = false$
   \ENDIF
   \ENDFOR
   \UNTIL{$noChange$ is $true$}
\end{algorithmic}
\end{algorithm}
 
\subsection{Tables} 
 
You may also want to include tables that summarize material. Like 
figures, these should be centered, legible, and numbered consecutively. 
However, place the title {\it above\/} the table with at least 
0.1~inches of space before the title and the same after it, as in 
Table~\ref{sample-table}. The table title should be set in 9~point 
type and centered unless it runs two or more lines, in which case it
should be flush left.

% Note use of \abovespace and \belowspace to get reasonable spacing 
% above and below tabular lines. 

\begin{table}[t]
\caption{Classification accuracies for naive Bayes and flexible 
Bayes on various data sets.}
\label{sample-table}
\vskip 0.15in
\begin{center}
\begin{small}
\begin{sc}
\begin{tabular}{lcccr}
\hline
\abovespace\belowspace
Data set & Naive & Flexible & Better? \\
\hline
\abovespace
Breast    & 95.9$\pm$ 0.2& 96.7$\pm$ 0.2& $\surd$ \\
Cleveland & 83.3$\pm$ 0.6& 80.0$\pm$ 0.6& $\times$\\
Glass2    & 61.9$\pm$ 1.4& 83.8$\pm$ 0.7& $\surd$ \\
Credit    & 74.8$\pm$ 0.5& 78.3$\pm$ 0.6&         \\
Horse     & 73.3$\pm$ 0.9& 69.7$\pm$ 1.0& $\times$\\
Meta      & 67.1$\pm$ 0.6& 76.5$\pm$ 0.5& $\surd$ \\
Pima      & 75.1$\pm$ 0.6& 73.9$\pm$ 0.5&         \\
\belowspace
Vehicle   & 44.9$\pm$ 0.6& 61.5$\pm$ 0.4& $\surd$ \\
\hline
\end{tabular}
\end{sc}
\end{small}
\end{center}
\vskip -0.1in
\end{table}

Tables contain textual material that can be typeset, as contrasted 
with figures, which contain graphical material that must be drawn. 
Specify the contents of each row and column in the table's topmost
row. Again, you may float tables to a column's top or bottom, and set
wide tables across both columns, but place two-column tables at the
top or bottom of the page.
 
\subsection{Citations and References} 

Please use APA reference format regardless of your formatter
or word processor. If you rely on the \LaTeX\/ bibliographic 
facility, use {\tt natbib.sty} and {\tt icml2015.bst} 
included in the style-file package to obtain this format.

Citations within the text should include the authors' last names and
year. If the authors' names are included in the sentence, place only
the year in parentheses, for example when referencing Arthur Samuel's
pioneering work \yrcite{Samuel59}. Otherwise place the entire
reference in parentheses with the authors and year separated by a
comma \cite{Samuel59}. List multiple references separated by
semicolons \cite{kearns89,Samuel59,mitchell80}. Use the `et~al.'
construct only for citations with three or more authors or after
listing all authors to a publication in an earlier reference \cite{MachineLearningI}.

Authors should cite their own work in the third person
in the initial version of their paper submitted for blind review.
Please refer to Section~\ref{author info} for detailed instructions on how to
cite your own papers.

Use an unnumbered first-level section heading for the references, and 
use a hanging indent style, with the first line of the reference flush
against the left margin and subsequent lines indented by 10 points. 
The references at the end of this document give examples for journal
articles \cite{Samuel59}, conference publications \cite{langley00}, book chapters \cite{Newell81}, books \cite{DudaHart2nd}, edited volumes \cite{MachineLearningI}, 
technical reports \cite{mitchell80}, and dissertations \cite{kearns89}. 

Alphabetize references by the surnames of the first authors, with
single author entries preceding multiple author entries. Order
references for the same authors by year of publication, with the
earliest first. Make sure that each reference includes all relevant
information (e.g., page numbers).

\subsection{Software and Data}

We strongly encourage the publication of software and data with the
camera-ready version of the paper whenever appropriate.  This can be
done by including a URL in the camera-ready copy.  However, do not
include URLs that reveal your institution or identity in your
submission for review.  Instead, provide an anonymous URL or upload
the material as ``Supplementary Material'' into the CMT reviewing
system.  Note that reviewers are not required to look a this material
when writing their review.


% Acknowledgements should only appear in the accepted version. 
\section*{Acknowledgments} 
 
\textbf{Do not} include acknowledgements in the initial version of
the paper submitted for blind review.

If a paper is accepted, the final camera-ready version can (and
probably should) include acknowledgements. In this case, please
place such acknowledgements in an unnumbered section at the
end of the paper. Typically, this will include thanks to reviewers
who gave useful comments, to colleagues who contributed to the ideas, 
and to funding agencies and corporate sponsors that provided financial 
support.  


% In the unusual situation where you want a paper to appear in the
% references without citing it in the main text, use \nocite
\nocite{langley00}

\bibliography{example_paper}
\bibliographystyle{icml2015}

\end{document} 


% This document was modified from the file originally made available by
% Pat Langley and Andrea Danyluk for ICML-2K. This version was
% created by Lise Getoor and Tobias Scheffer, it was slightly modified  
% from the 2010 version by Thorsten Joachims & Johannes Fuernkranz, 
% slightly modified from the 2009 version by Kiri Wagstaff and 
% Sam Roweis's 2008 version, which is slightly modified from 
% Prasad Tadepalli's 2007 version which is a lightly 
% changed version of the previous year's version by Andrew Moore, 
% which was in turn edited from those of Kristian Kersting and 
% Codrina Lauth. Alex Smola contributed to the algorithmic style files.  
